%%%%%%%%%%%%%%%%%%%%%%%%%%%%% Define Article %%%%%%%%%%%%%%%%%%%%%%%%%%%%%%%%%%
\documentclass{article}
%%%%%%%%%%%%%%%%%%%%%%%%%%%%%%%%%%%%%%%%%%%%%%%%%%%%%%%%%%%%%%%%%%%%%%%%%%%%%%%

%%%%%%%%%%%%%%%%%%%%%%%%%%%%% Using Packages %%%%%%%%%%%%%%%%%%%%%%%%%%%%%%%%%%
\usepackage{xcolor}
%\pagecolor[rgb]{0.1,0.1,0.1} % Set page background color to dark
%\color[rgb]{0.9,0.9,0.9} % Set text color to light

\usepackage{geometry}
\usepackage{graphicx}
\usepackage{amssymb}
\usepackage{amsmath}
\usepackage{amsthm}
\usepackage{empheq}
\usepackage{mdframed}
\usepackage{booktabs}
\usepackage{lipsum}
\usepackage{graphicx}
\usepackage{color}
\usepackage{psfrag}
\usepackage{pgfplots}
\usepackage{bm}
\usepackage[makeroom]{cancel}
%%%%%%%%%%%%%%%%%%%%%%%%%%%%%%%%%%%%%%%%%%%%%%%%%%%%%%%%%%%%%%%%%%%%%%%%%%%%%%%

% Other Settings
%%%%%%%%%%%%%%%%%%%%%%%%%% Page Setting %%%%%%%%%%%%%%%%%%%%%%%%%%%%%%%%%%%%%%%
\geometry{a4paper}

%%%%%%%%%%%%%%%%%%%%%%%%%% Define some useful colors %%%%%%%%%%%%%%%%%%%%%%%%%%
\definecolor{ocre}{RGB}{243,102,25}
\definecolor{mygray}{RGB}{243,243,244}
\definecolor{deepGreen}{RGB}{26,111,0}
\definecolor{shallowGreen}{RGB}{235,255,255}
\definecolor{deepBlue}{RGB}{61,124,222}
\definecolor{shallowBlue}{RGB}{235,249,255}
%%%%%%%%%%%%%%%%%%%%%%%%%%%%%%%%%%%%%%%%%%%%%%%%%%%%%%%%%%%%%%%%%%%%%%%%%%%%%%%

%%%%%%%%%%%%%%%%%%%%%%%%%% Define an orangebox command %%%%%%%%%%%%%%%%%%%%%%%%
\newcommand\orangebox[1]{\fcolorbox{ocre}{mygray}{\hspace{1em}#1\hspace{1em}}}
%%%%%%%%%%%%%%%%%%%%%%%%%%%%%%%%%%%%%%%%%%%%%%%%%%%%%%%%%%%%%%%%%%%%%%%%%%%%%%%

%%%%%%%%%%%%%%%%%%%%%%%%%%%% English Environments %%%%%%%%%%%%%%%%%%%%%%%%%%%%%
\newtheoremstyle{mytheoremstyle}{3pt}{3pt}{\normalfont}{0cm}{\rmfamily\bfseries}{}{1em}{{\color{black}\thmname{#1}~\thmnumber{#2}}\thmnote{\,--\,#3}}
\newtheoremstyle{myproblemstyle}{3pt}{3pt}{\normalfont}{0cm}{\rmfamily\bfseries}{}{1em}{{\color{black}\thmname{#1}~\thmnumber{#2}}\thmnote{\,--\,#3}}
\theoremstyle{mytheoremstyle}
\newmdtheoremenv[linewidth=1pt,backgroundcolor=shallowGreen,linecolor=deepGreen,leftmargin=0pt,innerleftmargin=20pt,innerrightmargin=20pt,]{theorem}{Theorem}[section]
\theoremstyle{mytheoremstyle}
\newmdtheoremenv[linewidth=1pt,backgroundcolor=shallowBlue,linecolor=deepBlue,leftmargin=0pt,innerleftmargin=20pt,innerrightmargin=20pt,]{definition}{Definition}[section]
\theoremstyle{myproblemstyle}
\newmdtheoremenv[linecolor=black,leftmargin=0pt,innerleftmargin=10pt,innerrightmargin=10pt,]{problem}{Problem}[section]
%%%%%%%%%%%%%%%%%%%%%%%%%%%%%%%%%%%%%%%%%%%%%%%%%%%%%%%%%%%%%%%%%%%%%%%%%%%%%%%

%%%%%%%%%%%%%%%%%%%%%%%%%%%%%%% Plotting Settings %%%%%%%%%%%%%%%%%%%%%%%%%%%%%
\usepgfplotslibrary{colorbrewer}
\pgfplotsset{width=8cm,compat=1.9}
%%%%%%%%%%%%%%%%%%%%%%%%%%%%%%%%%%%%%%%%%%%%%%%%%%%%%%%%%%%%%%%%%%%%%%%%%%%%%%%

%%%%%%%%%%%%%%%%%%%%%%%%%%%%%%% Title & Author %%%%%%%%%%%%%%%%%%%%%%%%%%%%%%%%
\title{AP Physics: Unit 1}
\author{Isaac Rich}
%%%%%%%%%%%%%%%%%%%%%%%%%%%%%%%%%%%%%%%%%%%%%%%%%%%%%%%%%%%%%%%%%%%%%%%%%%%%%%%

\begin{document}
\maketitle

\section{Scalars and Vectors}
Scalars and vectors are two fundamental types of quantities in physics and mathematics. Scalars only have magnitude (or size), while vectors have both magnitude and direction. In this section, we'll explore examples of scalars and vectors, including distance, displacement, speed, and velocity.

\section*{Examples}
\subsection{Distance (Scalar)}
If you walk 5 kilometers, the distance traveled is a scalar quantity because it only has magnitude (5 km) and no direction. The total path taken doesn't matter; only the magnitude of the motion is considered.

\subsection{Displacement (Vector)}
If you walk 5 kilometers north, the displacement is a vector quantity. It has both magnitude (5 km) and direction (north). Displacement is concerned with the change in position, taking into account the initial and final points.

\[
\text{Displacement} = \Delta x = x_f - x_0
\]
\[
x_f \text{ refers to the value of the final position}
\]
\[
x_0 \text{ refers to the value of the initial position}
\]
\[
\Delta x \text{ is the symbol used to represent displacement}
\]

\subsection{Speed (Scalar)}
If a car is moving at a speed of 60 miles per hour, the speed is a scalar quantity. It only indicates how fast the car is moving, without specifying the direction.

\subsection{Velocity (Vector)}
If a car is moving at 60 miles per hour eastward, the velocity is a vector quantity. Velocity has both magnitude (60 mph) and direction (east). It describes the rate of change of displacement with respect to time.
\newpage
\section{Reference Frames}
\paragraph{How the choice of reference frame is related to speed and velocity measurements}
\subsection{Example}
A reference frame is a like a fixed point. Properties of other objects such as: position, velocity etc.are measured using the point.
It is so because no point in the universe is stationary or static. Every point is moving depending on another `so called' static point.
See it like this: you are going to a amusement park in a bus with your friend. When the bus starts moving you see everything outside the bus going backwards. Here you are the reference frame. But for a person standing beside the road who has just missed the bus would `observe' your bus going onward with you and your friend. So for the pedestrian both you and your friend are moving at a certain speed. But for you, you see that your friend is just sitting beside you, according to you, he is not moving but stationary as you are.
So the summary is when you are the frame of reference you and your friend are stationary and the pedestrian is moving. For the pedestrian it is the vice versa.

\section{Calculating average velocity or speed}
Although speed and velocity are often words used interchangeably, in physics, they are distinct concepts. Velocity (v) is a vector quantity that measures displacement (or change in position, $\Delta {s} $) over the change in time ($\Delta {t} $), represented by the equation v = $\Delta {s}/ \Delta {t}  $. Speed (or rate, r) is a scalar quantity that measures the distance traveled (d) over the change in time ($\Delta {t}$), represented by the equation r = d/$\Delta {t}$.
\section{Solving for time}
Rate of change in position, or speed, is equal to distance traveled divided by time (r=d/t). To solve for time, divide the distance traveled by the rate. For example, if Cole drives his car 45 km per hour and travels a total of 225 km, then he traveled for 225/45 = 5 hours.
\section{Displacement from time and velocity}
If Maricia travels for 1 minute at 5 m/s to the south, how much will she be displaced?
\[
\text{Steps}
\]
\[
\overrightarrow{v} = \dfrac{\overrightarrow{s}}{\Delta{t}}
\]
\[
\overrightarrow{s} = \overrightarrow{v} * \Delta{t}
\]
\[
\overrightarrow{s} = 5\text{ m/s} \times 1\text{ min}
\]
\[
\overrightarrow{s} = 5 \times 60
\]
\[
\overrightarrow{s} = 300 \text{ meters to the south}
\]

\section{Acceleration: Change in velocity over time}
Acceleration (a) is the change in velocity ($\Delta{v}$) over the change in time ($\Delta{t}$), represented by the equation a = $\Delta{v}/\Delta{t}$. This allows you to measure how fast velocity changes in meters per second squared ($m/s^2$). Acceleration is also a vector quantity, so it includes both magnitude and direction.
\subsection{Examples}
Porsche 911 0--60 in 3 seconds (East)
\[
\dfrac{60-0\text{ mph }}{3\text{ seconds }}=20\dfrac{\text{miles}}{\text{hours}}/\text{second East}
\]
\[
20\dfrac{\text{miles}}{\cancel{\text{hour}}\times\text{seconds}}*\dfrac{1}{3600}\dfrac{\cancel{\text{hour}}}{\text{seconds}}=\dfrac{20}{3600}\dfrac{\text{miles}}{\text{second}^2}=\dfrac{1}{180}\dfrac{\text{miles}}{\text{sec}^2}=\dfrac{1}{180}\dfrac{\text{miles}}{\text{sec}}/\text{sec}
\]

\section{Deriving displacement as a function of time, acceleration, and initial velocity}
Displacement in physics is a vector quantity that measures the change in position of an object over a given time period. Learn how to calculate an object’s displacement as a function of time, constant acceleration and initial velocity.
\[
v_1 = 19.6m/s
\]
\[
a_g = -9.8m/s^2
\]
\[
\text{ force of gravity }: F = m*g
\]
\[
F = G\dfrac{m_1m_2}{r^2}
\]
\[
S = V_{avg} * \Delta{T}
\]
\[
\dfrac{v_i+v_f}{2}
\]
\[
\dfrac{v_i+v_f}{2} = \dfrac{v_i + \overbrace{v_i + a}^{v_f}*{\Delta{t}}}{2}
\]
\[
s=(\dfrac{2v_i}{2}+\dfrac{2*\Delta{t}}{2})\Delta{t}
\]
\[
s = v_i*\Delta{t}+1/2*a*\Delta(t^2)
\]
\[
d = v_{it}+\dfrac{1}{2}at^2
\]
A boat is stationary at 12 meters away from a dock. The boat then begins to move toward the dock with an acceleration of $5.0\dfrac{m}{2^2}$
\[
\begin{aligned}
x_{\text{f}} & = x_{\text{i}} + v_{\text{i}}t + \dfrac{1}{2}at^2 \\
0\, \text{m} & = 12\, \text{m} + (0\, \dfrac{\text{m}}{\text{s}})t + \dfrac{1}{2}(-5.0\, \dfrac{\text{m}}{\text{s}^2})t^2 \\
0\, \text{m} & = 12\, \text{m} - (2.5\, \dfrac{\text{m}}{\text{s}^2})t^2 \\
(2.5\, \dfrac{\text{m}}{\text{s}^2})t^2 & = 12\, \text{m}
\end{aligned}
\]
\[
\begin{aligned}
t^2 & = \dfrac{12}{2.5}\, \text{s}^2 \\
t   & = \sqrt{\dfrac{12}{2.5}}\, \text{s} \\
t   & \approx 2.2\, \text{s}
\end{aligned}
\]

\section{Plotting projectile displacement, acceleration, and velocity}

\end{document}
